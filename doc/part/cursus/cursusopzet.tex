\chapter{Cursusopzet}
Tijdens de cursus \textbf{Backend Programming 2 (BEP2)} leren we 
een structureel kwalitatief hoogstaande back-end 
op te zetten aan de hand van bestaande oplossingen en standaarden.
Hoewel we een boel zullen herhalen, bouwt dit voort 
op de kennis en vaardigheden
die we bij \textbf{Object-Oriented Analysis (OOAD)}, 
\textbf{Object-Oriented Programming (OOP)}
en \textbf{Backend Programming 1 (BEP1)} hebben opgedaan.

Daarnaast is er samenhang met \textbf{Data en Persistency (DNP)}:
we maken gebruik van een  relationele database om langdurige 
opslag te verwezenlijken. Bij DnP wordt meer stilgestaan bij 
de onderliggende techniek, terwijl we bij BEP2 dankbaar gebruik 
maken van library en frameworks die een hoop werk uit handen halen.

Deze cursus bereidt je voor op projecten en andere cursussen, zoals
\textbf{Continuous Integration and Software Quality (CISQ)} 
en \textbf{Software Architecture (SARCH)},
maar ook op je stages en je carrière. Hoewel we Java als taal gebruiken 
en Spring Boot als framework, zijn de ideeën namelijk
toepasbaar op heel veel andere programmeertalen en frameworks!
Let dus goed op als je binnenkort al wil starten als software developer.

De vraag die in deze cursus centraal staat is als volgt:

\begin{defbox}{Centrale vraag}
Hoe kunnen we een flexibele, onderhoudbare webapplicatie opzetten, waarin we bestaand werk kunnen hergebruiken?
\end{defbox}

\section{Kennisbasis}
Om deze centrale vraag te beantwoorden ontwikkelen we 
een web-applicatie met \textit{Spring Boot}, een veelgebruikt 
en modern Java-framework voor web development. Voor de persistentie 
gebruiken we \textit{PostgreSQL} tezamen met \textit{Spring JPA} dat
onderwater \textit{Hibernate} gebruikt als object-relational mapper.

We werken aan flexibele software door te letten
op een separation of concerns, loose coupling en high cohesion.
Daar kan een modulaire, gelaagde applicatiearchitectuur aan bijdragen.

Voor het ontwerp en de realisatie van onze software benutten we 
het objectmodel: object oriëntatie zoals het bedoeld is! 
Dit wordt aangevuld door een aantal design principles om tot een 
beter ontwerp te komen. We zetten polymorfisme en compositie in 
om via dependency injection de onderdelen binnen onze applicatie 
op flexibele wijze los te koppelen van elkaar.

In de realisatie van het project implementeren we een aantal
design patterns, een aantal standaardoplossingen voor 
veelvoorkomende problemen.

\newpage
\section{Leerdoelen}
Na deze cursus kan de student:
\begin{itemize}
    \item de structurele kwaliteit van software beoordelen aan de hand van de begrippen \textbf{separation of concerns}, \textbf{cohesion}, \textbf{coupling}
    \item de eigenschappen van object-orientatie, in de zin van het \textbf{objectmodel}, toepassen om onderhoudbare, object-georiënteerde software te ontwerpen en te realiseren
    \item \textbf{dependency injection} toepassen om flexibele, losgekoppelde software te verwezenlijken, waarvan de onderdelen te vervangen zijn
    \item \textbf{design principles} toepassen, zoals SOLID, om object-georiënteerde software te ontwerpen en realiseren
    \item \textbf{design patterns} herkennen en realiseren, zoals die van de Gang of Four, als abstracte standaardoplossingen om software te ontwerpen en te realiseren
    \item de \textbf{REST-principes} op de juiste wijze toepassen op een web applicatie
    \item een \textbf{web framework}, zoals Spring Boot, inzetten om een REST API te realiseren
    \item een \textbf{object-relational mapper}, zoals Hibernate, inzetten om object-geörienteerde persistentie vorm te geven
    \item werken binnen een \textbf{gelaagde architectuur}, waarbij elke laag ziet op één abstractieniveau en één soort logica
\end{itemize}

\section{Toetsing}
De leerdoelen worden getoetst aan de hand van een \textbf{individuele opdracht met interview}.
De criteria hiervoor zijn opgenomen op Canvas. 

Dit betekent dat de student gedurende de cursus een softwareproject realiseert.
Aan het eind van de cursus wordt een assessmentmoment ingepland. De student
presenteert zijn of haar werk en beantwoordt aan de hand van het project
vragen over de theorie. Ter voorbereiding is het verstandig om de theorie al 
een beetje in de presentatie op te nemen, 
bijvoorbeeld middels een korte PowerPoint-presentatie.

Studenten hebben laten zien dat het goed mogelijk is om een mooi cijfer te halen 
door naar deze criteria te kijken, de colleges bij te wonen en tijdens de cursus 
feedback te vragen en te verwerken. Het loont om daar de tijd voor te nemen.

\subsection{Studiebelasting}
Voor deze cursus staat een studie belasting van 5 ECs (European Credits). 
In Nederland wordt dit gelijkgesteld met 140 uur inclusief colleges.
De opdracht is echter zo ontworpen dat er voldoende ruimte is.
Zo kan je weggezakte kennis ophalen, zaken opzoeken en uitproberen
en feedback verwerken.

Het loont om een weekplanning te maken om regelmatig aan de opdracht 
te werken. Tijdens het programmeren zelf kan je een stuk sneller werken door 
gebruik te maken van alle shortcuts en functionaliteiten die je 
IDE (bij voorkeur IntelliJ IDEA) biedt!

\subsection{Plagiaat}
Het is soms nuttig om even met een medestudent te overleggen,
maar pas op dat je niet hele stukken code van elkaar overneemt.
Daar leer je weinig van. Bovendien kan je je schuldig maken aan 
plagiaat. Dit houdt in dat je teveel van andermans werk overneemt 
of werk aanhaalt zonder bronaanduiding. Het kan gaan om het kopiëren 
van werk van een medestudent, maar ook om andere bronnen zoals die op 
internet te vinden zijn. Mocht je toch code overnemen van internet 
zet de vindplaats er dan bij. Een URL met een comment is voldoende.

Vermoedens van plagiaat worden gemeld bij de examencommissie. 
Als na onderzoek blijkt dat er dat er sprake is
van plagiaat, kan er een (zware!) maatregel opgelegd worden.
Zie voor details het onderwijs- en examenreglement (OER) en de studiegids.

\section{Begeleiding en ondersteuning}
Tijdens de cursus leggen docenten de lesstof uit en geven zij voorbeelden.
Ook is er ruimte voor extra toelichting en om samen te programmeren.

Docenten geven individuele of groepsgebaseerde feedback
op basis van het ingeleverde werk. Commit en push je werk dus regelmatig
op GitHub. Feedback helpt om efficiënt te leren, 
maar het is natuurlijk nog effectiever om ook zelf vragen voor te bereiden! 

Trek zo snel mogelijk aan de bel als je ergens niet uitkomt of 
je qua tijdsplanning in de problemen komt.
Software development is af en toe overweldigend en soms kan een 
klein zetje al genoeg zijn om verder te komen. Het is ook niet erg
als je wat meer toelichting of instructie zou willen. Geef dit aan.
Docenten en medestudenten willen je graag verder helpen!
Samen maken we er een mooie cursus van.