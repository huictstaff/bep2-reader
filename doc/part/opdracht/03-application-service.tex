\chapter{Opdracht 3: Application service}
We hebben het project aan de praat, de eisen geanalyseerd, 
de use cases gemodelleerd en de eerste domeinconcepten gerealiseerd. 
Het doet allemaal alleen nog niet zoveel!

We gaan nu bezig met de kern van onze applicatie:
use cases en domeinacties. Deze opdracht duurt waarschijnlijk het langst!
Het is niet erg om te beginnen aan deze opdracht en hem later te
verbeteren wanneer we de controller en de repository werkend hebben.

\section{Spring Boot}
Spring Boot is een veelgebruikte standaardconfiguratie van het Spring framework.

\subsection{Frameworks}
Een \textit{framework} is code van anderen die je een hoop werk uit handen 
nemen ten aanzien van een of meer functionaliteiten. Het is een concrete, herbruikbare standaardoplossing.
Kenmerkend aan een framework is dat je als developer een deel van de controle 
uit handen geeft. Er is sprake van \textit{inversion of control}: het framework 
roept onze code aan op bepaalde plekken. Interactie met het framework vindt op twee plaatsen 
plaats:
\begin{enumerate}
    \item Entry points: hier roepen we het framework aan
    \item Hot spots: hier haken we onze eigen code in het framework en roept het framework onze code aan
\end{enumerate}

Er zijn twee soorten hot spots, die je kan herkennen vanuit het objectmodel:
\begin{enumerate}
    \item Composition hot spots: integratie van het aan een interface meegeven 
    van implementerende dependencies
    \item Inheritance hot spots: integratie via het overerven 
    van een vooropgezette klassenstructuur
\end{enumerate}

Beide vormen komen vaak voor in een framework.

In Spring herkennen we de \texttt{CasinoApplication} als entry point.
Deze is geannoteerd met \texttt{@SpringBootApplication}. Vervolgens laadt
Spring allerlei configuratie in, voert het een componentscan uit en klikt het dependencies in elkaar.
Spring's dependency injection is één van de composition hot spots die te vinden zijn in Spring,
terwijl Spring's repositories een vorm zijn inheritance hot spots.

\subsection{Dependency injection}
Dependency injection is niets anders dan het meegeven van afhankelijkheden,
in plaats van ze binnen de klasse aan te maken. Dit zorgt ervoor dat een klasse 
makkelijker uitbreidbaar en testbaar is. Hierover in een latere cursus meer.

In Spring maken we services aan door de klasse te annoteren met de \texttt{@Service} annotatie. 
We kunnen met dependency injection werken door de services (en opslagmechanismen) waarvan we afhankelijk zijn 
aan te geven in de constructor. Spring vindt dan automatisch welke services we bedoelen. 
Dit heet \textit{autowiring}.
Spring gaat tijdens een opstarten van de applicatie door onze code op zoek naar 
\texttt{@Bean}, \texttt{@Component}, \texttt{@Service}. Dit noem je Beans. 
Spring kijkt voor de dependencies naar de constructor en de benodigde interfaces 
en kijkt of er Beans zijn geconfigureerd die aan die interface voldoen. Met autowiring 
injecteert Spring de benodigde afhankelijkheid dus automatisch!

Kijk voor een voorbeeld wederom in het Chips-component.

Als alternatief voor constructor-injection kan 
je ook werken met de \texttt{@Autowired} annotatie 
op setters, constructor parameters of properties. 

Als we meerdere implementaties hebben van dezelfde interface 
(bijvoorbeeld omdat we een testimplementatie hebben), 
dan moeten we specifiek aangeven welke service we willen gebruiken. 
Dat kunnen we doen met de \texttt{@Qualifier} annotatie op zowel de service als binnen de constructor. 
Daarmee kwalificeren we om welke implementatieklasse het gaat door 
dit aan te geven boven elke \texttt{@Service} en vóór elke parameter in de 
constructor die de service gebruikt.

Met \texttt{@Value} kan je aangeven dat een waarde moet worden geïnjecteerd,
afkomstig is uit configuratie.

\section{Libraries}
Een library is ook een voorbeeld van een concrete, herbruikbare standaardoplossing.
Het belangrijkste verschil tussen een library en een framework is dat 
een framework veel meer controle opeist. Een library wordt door onze code aangeroepen
terwijl een framework uiteindelijk onze code aanroept (inversion of control).

Wel zie je vaak dat een framework gebruik maakt of zich openstelt voor verschillende
libraries door middel van composition hot spots. Vaak heb je dan een adapter nodig 
om de interface van het framework aan te passen aan de geboden facade van de library.
Dit zijn voorbeelden van design patterns. Hier gaan we het later uitgebreid over hebben.

\section{Stap 1: Maak een applicatieservice}
In ons domein hebben we een aantal domeinklassen gemaakt die verantwoordelijk 
zullen worden voor kleine acties per concept. Deze worden aan elkaar geknoopt 
door een dienstverlenend object dat verantwoordelijk is voor het omzetten van 
use cases naar domeinacties met behulp van infrastructuur. 
Dat wordt ook wel een \textit{application service} genoemd.

Maak eerst weer een package aan onder onze blackjack-component met de naam \texttt{application}
Dit is de applicatielaag waarin we onze taakgerichte logica (use cases) gaan uitvoeren.
Maak in die package een klasse aan met de naam \texttt{BlackjackService}.
Het is immers een dienst die de acties aanbiedt om blackjack te kunnen spelen!
Deze wordt uiteindelijk aangesproken door de controller.

Vervolgens moeten we aan Spring doorgeven dat hij deze service kan gebruiken 
om te meegeven aan een andere service of bijvoorbeeld een controller als deze 
deze nodig heeft in de controller. Spring bevat een automatisch mechanisme 
voor \textit{dependency injection} en gebruikt daar annotaties voor. Om de service 
als zodanig vindbaar te maken, kan je \texttt{@Service} boven de klassedeclaratie plaatsen.

Dit zou er als volgt uit moeten zien:
\begin{minted}{java}
package nl.hu.bep2.casino.blackjack.application;

@Service
public class BlackjackService {
}
\end{minted}

\section{Stap 2: Bedenk welke methodes we nodig hebben}
Welke acties moet onze service aanbieden? 
Dit komt overeen met de use cases van onze component! 
Als het goed is, hebben we hiervoor een use case diagram gemaakt.

Maak public methods aan met de namen van deze use cases volgens 
de naming conventions van Java (\textit{camelCase}). 
Bedenk per methode ook wat voor een input we nodig hebben.
Voor sommige methoden is het handig om het spel te kunnen identificeren
op basis van een id (type: \textit{Long}). Voor de meeste methoden hebben 
we daarnaast de naam van de gebruiker nodig (type: \textit{String}).

De identifier zullen we automatisch door de database laten genereren
zodra we met persistentie bezig gaan.

\subsection{Voorbeeld: Het starten van het spel}
Laten we als voorbeeld het starten van een spel als use case nemen.
Het is belangrijk om hier een duidelijke, beschrijvende naam voor te pakken,
bijvoorbeeld: \textit{startGame} of \textit{start}.

\subsubsection{Method arguments}
Wat moeten we allemaal meegeven als parameters om het spel te starten?

Om het spel te kunnen opzoeken voor deze speler is het handig om de naam van de speler  
te bewaren. We kunnen een parameter toevoegen met als type \texttt{String} en 
als naam \texttt{playerName}. 
Omdat een speler meerdere spellen kan hebben, moeten we ervoor zorgen 
dat de database ook elk spel voorziet van een unieke identifier. 
Dat doen we later wanneer we met persistentie bezig gaan.

Het is handig om het spel meteen te beginnen met een inleg. Dat bespaart ons een
extra HTTP request! Laten we de parameter \texttt{Integer bet} toevoegen.

Meer hebben we niet nodig van de speler!

\subsubsection{Return type}
Wat willen we teruggeven nadat we het spel hebben gestart? 
Hier kunnen we wat slims verzinnen om aan de speler te laten zien hoe 
het spel er nu uitziet: de spelvoortgang. 
We willen uiteindelijk namelijk in de controller
een bericht terugsturen naar de HTTP client met daarin een JSON body 
met alle informatie. Een front-endprogrammeur kan dat mooi weergeven met
afbeeldingen en interacties.

Welke informatie willen we aan de speler laten zien
en hoe kunnen we dat het best structureren? Dat laten we aan jou over!
Alvast een tip: we kunnen een String teruggeven, 
maar dan gaat er een boel gestructureerde informatie verloren!

\subsection{Stap 3: Vul de methodes in}
Nadat we bedacht hebben welke methodes onze service moet hebben, 
kunnen we naar de invulling van die methodes kijken.
Het is het mooiste als applicatieservices niet teveel logica bevatten,
maar het gros van het werk door domeinobjecten wordt gedaan.
Op deze manier houd je een abstract en herbruikbaar domein en is je 
applicatieservice een soort samenvatting van hoe het domein zich gedraagt.

Probeer in algemene bewoordingen de acties te beschrijven 
en gaandeweg acties toe te voegen aan de centrale domeinentiteit
en de domeinobjecten die daarin gebruikt worden.

\subsubsection{Voorbeeld: Method body van startGame}
We hebben een methodenaam, parameters en een return type gedeclareerd.
Welke stappen willen we uitvoeren in de methode? 

Globaal zal je op de volgende zaken uitkomen:

\begin{enumerate}
    \item Neem het aantal chips op ter hoogte van de bet
    \item Maak een nieuw spel aan
    \item Start het spel
    \item Sla het spel op
    \item Stort chips als sprake van blackjack of push
    \item Geef de voortgang terug
\end{enumerate}

Voor het starten van het spel kan je een methode op het Game object maken,
genaamd \textit{start}. Je kan ervoor kiezen om de benodigde parameters mee te geven aan de 
constructor of de start methode. Wat deze methode moet doen laten we aan jou over.
Een aantal tips: wat moet de speltoestand zijn bij het beginnen van het spel?
Laat je het spel zelf een nieuwe Deck maken of doen we dat in de application service?

Bij de uitvoering van \textit{game.start()}, maken we onder andere gebruik van 
Deck en vast ook wel van andere klassen en enums! 
De kaarten moeten worden geschud en op de hand worden gebracht van de speler 
en van de dealer. Vervolgens moeten scores berekend worden en de huidige speltoestand 
geupdatet worden.

Het opslaan van het spel kan je nog even laten zitten,
maar het is misschien handiger om dit tijdelijk te doen 
door een field op te nemen in de application service.
Je zou hiervoor een \texttt{Map<String, Game>} kunnen gebruiken 
waarmee tijdelijk spellen (\textit{value}) bewaard kunnen worden 
op basis van spelernaam (\textit{key}).

Dit gaan we in een van de volgende opdrachten vervangen met apart object dat verantwoordelijk 
is voor de langdurige opslag van spellen: een GameRepository.

\subsubsection{Bescherm in domeinacties tegen ongeldige situaties (invariants)}
Je domein bepaalt hoe de regels van de kernconcepten eruit zien.
Bij een spel zijn het bijvoorbeeld letterlijk de spelregels.
Zie bijvoorbeeld de Chips-klasse: je mag geen negatieve hoeveelheid opnemen 
en je mag geen chips opnemen als je saldo te laag is.

\begin{minted}{java}
 public void withdraw(Long amountToWithdraw) {
    if (amountToWithdraw < 0) {
        throw new NegativeNumberException("Cannot withdraw a negative amount: " + amountToWithdraw);
    }

    long newAmount = this.amount - amountToWithdraw;
    if (newAmount < 0) {
        throw new NotEnoughChipsException(
                String.format(
                        "Cannot withdraw %d chips: %d chips remaining",
                        amountToWithdraw,
                        this.amount
                )
        );
    }

    this.amount = newAmount;
}    
\end{minted}

Voor het spelen van het spel geldt hetzelfde. Je mag natuurlijk 
geen zetten doen als het spel is afgelopen! Hiervoor kan je een \textit{guard-clause} gebruiken:
een \textit{if-statement} die een exception gooit als iets niet mag. Je hebt geen \textit{else} nodig! 
De flow wordt immers doorbroken als er sprake is van een exception.

\subsubsection{Overige use cases en domeinacties}
Doe hetzelfde voor de overige use cases en domeinacties. 
Hier gaat een boel tijd inzitten, 
dus het is niet erg als je dit later verbetert wanneer we 
de persistentie en de web API ingericht hebben.

Commit je wijzigingen met een duidelijke naam, 
bijvoorbeeld: "Add use cases in blackjack service". 
Push de wijzigingen naar je remote GitHub repository.