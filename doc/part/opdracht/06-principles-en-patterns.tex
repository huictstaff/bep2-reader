\chapter{Opdracht 6: Design principles en design patterns}
Begin aan deze opdracht zodra je wat verder bent 
met je domein, je applicatieservice, je controller en je opslag.
Vergeet niet je werk te committen en te pushen.

We gaan nu namelijk kijken hoe we ons project kunnen verbeteren aan de hand van 
een aantal ontwerpprincipes (design principles) en standaardoplossingen (design patterns).
Design principles en design patterns worden in de praktijk ingezet om 
de structuur van software meer onderhoudbaar te maken. Vaak kan met een aantal kleine 
aanwijzingen ervoor gezorgd worden dat de samenhang wordt vergroot (\textit{high cohesion})
of dat modules meer losgekoppeld worden (\textit{loose coupling}).

\section{Design principles}
Hier volgt een samenvatting van de belangrijkste ontwerpprincipes die je in de praktijk 
zult tegenkomen en waar je in deze cursus vragen over kunt verwachten.

\subsection{Gang of Four-principles: ICE}
De Gang of Four (Gamma e.a, 1996) haalt een aantal principes aan 
in hun invloedrijke werk `Design Patterns' die volgens hen tot onderhoudbare 
software leiden. Deze principes zijn leidend geweest voor het bedenken 
van een aantal design patterns: standaardpatronen die een oplossing bieden 
voor bepaalde onderhoudsproblemen.

Deze design principles worden meer uitgebreid behandeld tijdens de colleges,
maar hier hebben we een samenvatting opgenomen. Voor meer informatie zou 
je het boek Design Patterns kunnen raadplegen.

\subsubsection{Program to an Interface}
Bij het object-georiënteerd programmeren werken we met abstracties (zie: Objectmodel van Booch).
We proberen de wereld te modelleren in termen klassen, interfaces, enums en packages.
Een klasse staat dan voor een bepaald concept waar we toestand en gedrag aan willen 
toekennen. Denk bijvoorbeeld aan een blackjack-potje, een hand met kaarten of een deck.

Het gedrag dat we naar buiten toe kenbaar maken wordt ook wel de 
application programming interface (API) van die klasse genoemd.
In Java kunnen we deze afdwingen door een \texttt{interface} te implementeren.
Het mooie aan het werken met abstracties is dat je als gebruiker niet bekend hoeft 
te zijn met de interne werking wanneer je er gebruik van wil maken.
Het principe \textit{Program to an Interface} 
ziet erop dat je als gebruiker van een object of 
klasse zoveel mogelijk afhankelijk wil zijn van diens 
API of interface en niet van de details van 
de daarachtergelegen implementatie.

Wanneer een service bijvoorbeeld een domeinactie wil uitvoeren op een blackjack-potje, zoals het 
verwerken van een \textit{start game} of een \textit{hit}, 
hoeft de service geen weet te hebben welke gevolgen dat heeft voor 
de deck, de hand, de status van het spel of de bet. 
Dat weet het blackjack-potje zelf af te handelen!

\subsubsection{Favor Composition over Inheritance}
In veel object-georiënteerde systemen wordt implementation inheritance
gebruikt als belangrijkste mechanisme voor hergebruik. Dit heeft echter
twee belangrijke nadelen: je introduceert coupling tussen parents en children 
(en indirect tussen siblings) en je moet al vroegtijdig rekening houden met 
generieke abstracties. Dit laatste is vaak lastig, omdat je nog niet bekend 
bent met wat je domein allemaal doet. Dit leidt vaak tot grote klassen met 
een hoop if-statements.

Een handigere manier om hergebruik te realiseren is door klassen 
te maken die zich enkel richten op één kernfunctionaliteit.
Wanneer je die functionaliteit wil hergebruiken in een bepaalde
andere klasse, dan kan je deze samenstellen uit onder andere de klasse die 
de her te gebruiken functionaliteit bevat. Dat is de kern van 
\textit{Favor Composition over Inheritance}.
Vaak gebruik je hiervoor dependency injection:
we geven de benodigde functionaliteit mee aan de constructor of een setter
van de samengestelde klasse.

Overigens betekent dit niet dat je nóóit implementation inheritance mag gebruiken.
Zodra je meer weet over de abstracties in je project, weet je welke zaken je 
met inheritance kunt inrichten. Probeer dan wel de voorkeur te geven aan abstract base classes
en beperk de grote van de overervingsboom. Dit komt loose coupling ten goede.

\subsubsection{Encapsulate what Varies}
Je kan alleen composities maken als de code voldoende is opgebroken
in klassen die zich richten op één doel. In een aantal gevallen 
is het duidelijk wat in één klasse hoort te zitten, bijvoorbeeld
wanneer het gaat om een sluitende abstractie uit de echte wereld.
Denk bijvoorbeeld aan een speelkaart of een deck.

Vaak is het echter lastig te bepalen hoe je de rest van de code opbreekt. 
Een vuistregel om te bepalen welke stukken code in een bepaalde module gaan 
is door te kijken op welke punten variatie mogelijk moet zijn. Die dingen 
kan je samenvoegen: \textit{Encapsulate what Varies}. Om die variatie 
te realiseren zou je er ook een \textit{interface} tussen kunnen plaatsen,
zodat de implementatie dankzij polymorfisme kan variëren.

\subsection{SOLID-principles}
De SOLID-principles zijn object-georiënteerde ontwerpprincipes die door 
verschillende belangrijke software engineers zijn beschreven.
Ze zijn over de jaren heen verzameld door Robert C. Martin, 
welke ze heeft beschreven in boeken als 
`Agile Principles, Practices and Patterns' (2006), 
`Clean Code' (2008) en `Clean Architecture' (2017).

Deze principes komen vaak ter sprake tijdens het 
dagelijkse werk van een software developer. Daarom zie 
je discussies hierover vaak ook terugkomen tijdens 
sollicitatiegesprekken.

Hier volgt een korte samenvatting van deze principes.
Deze worden uitgebreid tijdens de colleges behandeld,
maar meer informatie is te vinden in bovenstaande boeken 
of door te zoeken op YouTube of via Google.

\subsubsection{Single Responsibility Principle (SRP)}
Een module moet zich richten op één verantwoordelijkheid.
Dit principe wordt vaak anders geformuleerd als 
`een module moet slechts een reden hebben om te wijzigen'.

Dit principe volgt eigenlijk al uit de \textit{separation of concerns}
en het vereisen van \textit{high cohesion}. Ook sluit het naadloos aan 
op \textit{Encapsulate what Varies}. Toch wordt het ontzettend
veel aangehaald.

Let wel dat het lastig kan zijn om te bepalen wat één
verantwoordelijkheid precies is. Het is aan software engineers om daar een 
goede balans in te vinden tussen handige abstracties en veel losse stukjes.
Vaak kom je al een heel eind door methoden op te breken in verschillende 
(private) methoden met een duidelijke naam.

\subsubsection{Open Closed Principle (OCP)}
Dit principe, geformuleerd door Bertrand Meyer, 
is een van de meest onbegrepen principes in 
de softwareindustrie.
Het luidt als volgt: ``Modules should be 
\textbf{open for extension, but closed for modification}''.

Dit houdt in dat je een klasse op zo'n manier moet ontwerpen
dat deze niet of nauwelijks intern gewijzigd hoeft te worden 
als je functionaliteit wil toevoegen. Dit doe je door het mogelijk 
te maken functionaliteit uit te wisselen door samenstellingen te 
maken van andere klassen.

Het lijkt dus op een combinatie van de ICE-principes!
Wees afhankelijk van abstracties zodat je gedrag van klassen
kunt uitbreiden door dit van buitenaf mee te geven.
Dit leidt tot flexibiliteit dankzij compositie en 
polymorfisme.

\subsubsection{Liskov Substitution Principle (LSP)}
Barbara Liskov heeft dit principe geformuleerd in de context 
van het overerven van klassen. ``Subtypes moeten kunnen worden 
vervangen door hun supertypes.'' Vaak wordt dit ook uitgelegd 
als dat child-klassen elkaar zouden moeten kunnen vervangen als 
ze dezelfde parent hebben.

De eisen die gesteld kunnen worden aan een methode, bijvoorbeeld 
ten aanzien van diens argumenten en return-waardes moeten voor 
elke subklasse vergelijkbaar zijn. Wanneer een klasse immers 
afhankelijk is van een supertype,
moet het geen aannames of checks hoeven doen op het specifieke 
subtype dat eraan wordt meegegeven.

\subsubsection{Interface Segregation Principle (ISP)}
`Clients should not be forced to depend on methods 
that they do not use.'

Het moet niet zo zijn dat implementerende klassen 
gedwongen worden om bepaalde methodes aan te bieden 
alleen maar om aan een interface of abstracte klasse te
voldoen. 

Denk bijvoorbeeld aan een \textit{BookStorage-interface}
die een methode vereist \textit{connectToDatabase}. Wat moet een
\textit{FileBookStorage-implementatie} dan aanbieden als implementatie
voor die methode? 
In zo'n geval heb je te maken met een \textit{leaky abstraction}:
het connecten met een database is immers alleen van 
toepassing op \textit{BookStorages} die met een database werken!
In de praktijk zou je dan een DatabaseConnection-klasse maken en die 
meegeven aan bijvoorbeeld de constructor van de \textit{PostgresqlBookStorage}.

\subsubsection{Dependency Inversion Principle (DIP)}
Modules moeten zoveel mogelijk afhankelijk zijn van abstracties
en niet van concrete implementatiedetails. 

Binnen een architectuur 
wordt dit ook wel uitgelegd als volgt. Je kernabstracties (je domein) moeten
zo min mogelijk afhankelijk zijn van technische implementatiedetails 
(je opslag- of presentatiemechanisme). Andersom is vaak geen probleem.
Dit bereik je door een abstractielaag tussen de concrete afhankelijkheid 
en de gebruiker ervan te stoppen. Vaak los je dit op met een abstracte klasse of interface.
Dit principe heeft dus veel weg van \textit{Program to an Interface}.
Dit zie je terug in patterns als 
\textit{facade}, \textit{gateway}, \textit{strategy} en \textit{adapter}.

\section{Stap 1: Evalueer en verbeter je project met design principles}
Omdat je wordt beoordeeld op de netheid en structuur van je software,
is het de moeite waard om voor jezelf je code te beoordelen aan de hand 
van design principles. Je kan hier ook vragen over verwachten tijdens 
het assessment.

Wil je hier op voorbereid zijn?
Vat voor jezelf samen in een markdown-bestand (bijvoorbeeld: `design-principles.md') 
of presentatie samen per design principle in hoeverre dat van toepassing is op je project en op 
welke manier je de toepassing ervan terugziet. 
Denk na over mogelijke verbeteringen en pas deze toe.
Probeer te kijken naar de design principles binnen je code
en leg de relatie met separation of concerns, high cohesion en loose coupling.
Hoe helpt het object model hierin? 

Ook kan je denken aan een aantal standaardverbeteringen:
naamgeving, code style en het verminderen van herhaling
(bijvoorbeeld door een klasse of (private) method te introduceren).
Een andere veelvoorkomende verbetering is het verminderen van static calls.
Wanneer een object voldoet aan ICE en SOLID (en high cohesion) zit 
in het object al de informatie die nodig is om een actie uit te voeren.

Heb je code verbeterd? Vergeet dan niet te committen en te pushen. Gebruik een 
beschrijvende message, zoals: `Refactor code according to design principles'.

\newpage
\section{Design patterns}
Waar libraries en frameworks concrete standaardoplossingen zijn die als 
\textit{third-party dependencies} in je project kunnen worden opgenomen,
zijn design patterns meer conceptuele standaardoplossingen. Het gaat om 
een soort standaardstructuur of -aanpak die gehanteerd kan worden bij 
veelvoorkomende problemen binnen een bepaalde context, gelet op bepaalde 
vereisten.

Hoe kan je bijvoorbeeld het creëren van instanties van verschillende implementaties van 
een interface netjes variëren op één plek? 
Dat kan door een \textit{creational pattern} te gebruiken,
zoals een \textit{Factory} of een \textit{Builder}.

\subsection{Een globaal overzicht van design patterns}
Hier volgt een globaal overzicht van de design patterns die in deze 
cursus aan bod komen. Dit is een selectie van bekende standaardoplossingen. 
Neem voor inspiratie en onderbouwing de slides door 
en eventueel het materiaal van \url{refactoring.guru} en \url{sourcemaking.com}.
De meeste patterns komen uit het Gang of Four-boek genaamd `Design Patterns',
een aantal anderen komen uit het boek `Patterns of Enterprise Application Architecture' van 
Martin Fowler. Alle patterns worden in meer of mindere mate toegepast in de praktijk.

Let op dat elke pattern een bepaalde structuur verwacht.
\textbf{Zorg er dus voor dat je zelf op basis van de slides
en de materialen hierboven weet hoe elke pattern er 
ongeveer werkt en in elkaar zit!}

\subsubsection{Creational patterns}
Inrichten creatie en levensduur van objecten
gelet op cohesion en coupling.

\begin{itemize}
    \item \textbf{Factory}: 
    (Sub)klassen of implementaties instantiëren tijdens runtime,
     waarbij de klassen en parameters kunnen variëren
    \item \textbf{Builder}: 
    Stapje voor stapje een complex object bouwen, 
    waarbij de stapjes en het eindproduct kunnen variëren
    \item \textbf{Singleton}:
    Een klasse die slechts één globale instantie aanbiedt 
    (gebruik spaarzaam)
\end{itemize}

\subsubsection{Structural patterns}
Structureren van objecten en deelsystemen
gelet op cohesion en coupling.

\begin{itemize}
    \item \textbf{Facade}:
    Maak een deelsysteem op eenduidige wijze 
    benaderbaar door een tussenliggende service 
    (met versimpelde interface) in te richten
    \item \textbf{Gateway (incl. DAO, Repository)}:
    Maak een extern systeem op eenduidige wijze 
    benaderbaar door een tussenliggende 
    (abstracte) service in te richten
    \item \textbf{Adapter}:
    Laat een klasse die imcompatibel is met 
    een interface toch deze invullen 
    door deze te verpakken in een klasse 
    die wel compatibel is met deze interface
    \item \textbf{Decorator}:
    Voegt nieuw gedrag toe aan een klasse 
    door deze in te pakken in een klasse 
    met dezelfde interface
    \item \textbf{Data Transfer Object (DTO)}:
    Groepeer data in één object zonder gedrag. 
    Vaak gebruikt om een bericht te modelleren
    tussen (deel)systemen.

\end{itemize}

\subsubsection{Behavioral patterns}
Efficiënt inrichten samenwerking tussen objecten
gelet op cohesion en coupling.

\begin{itemize}
    \item \textbf{Template Method}:
    Bied een abstract algoritme aan, 
    waarbij de volgorde van stappen vastligt 
    maar de implementatie van stappen kan 
    worden ingevuld door subklassen.
    \item \textbf{Strategy}:
    Encapsuleer runtime variatie van gedrag 
    waarvan een client afhankelijk is.
    \item \textbf{State}:
    Encapsuleer runtime variatie van toestanden 
    met complexe beslissingen.
    \item \textbf{Observer}:
    Wanneer één object verandert, 
    worden geïnteresseerden van deze 
    wijziging op de hoogte gebracht en geüpdatet.
\end{itemize}

\section{Stap 2: Herken toegepaste design patterns}
We zijn een heel eind gekomen met dit project. Onderweg hebben we stiekem al 
een aantal design patterns ingezet! Je kan tijdens het assessment de vraag 
verwachten welke design patterns je hebt ingezet en waar. 

Schrijf daarom in een markdown-bestand (bijvoorbeeld `design-patterns.md')
of in een presentatie waar je welke design patterns hebt opgenomen in de blackjack-component. 
Neem hiervoor de slides door en eventueel het materiaal van 
\url{refactoring.guru} en \url{sourcemaking.com}.

\newpage
\section{Stap 3: Voeg een of meer extra design pattern(s) toe}
Het herkennen van een design pattern is natuurlijk handig, maar je leert veel meer 
als je zelf een design pattern kunt realiseren binnen de blackjack-component!
Dat is de bedoeling van deze laatste stap voor het assessment.

Kies een of meer van de opties uit die hieronder zijn opgenomen en voeg de 
betreffende functionaliteit toe door een design pattern te gebruiken. 
Voeg deze patterns toe aan je markdown-bestand of presentatie.
Neem voor inspiratie en onderbouwing de slides door 
en eventueel het materiaal van \url{refactoring.guru} en \url{sourcemaking.com}.

Om een of meer patterns te implementeren kan je kiezen uit de volgende opties.
Je mag zelf bedenken hoe je dit precies uitwerkt en wat de randvoorwaarden zijn,
mits het voldoet aan de architectuur, design principles en design patterns.
Zo kan je ervoor kiezen om te werken met voorgeconfigureerde waarden waar de speler
op een of andere manier uit kan kiezen.
Denk ook goed na over wat de speler (client) door moet geven via de controller,
op welke plek dat moet gebeuren en of en met welke informatie de use case(s) 
uitgebreid moeten worden.

\begin{itemize}
    \item Optie A: Speler kan het aantal decks kiezen om mee te spelen
    \item Optie B: Speler kan zelf de kaarten binnen een decks samenstellen
    \item Optie C: Speler kan de uitbetalingsregels kiezen
    \item Optie D: Speler kan de scoreregels kiezen voor speelkaarten of handen
    \item Optie E: De spelregels kunnen verschillen op basis van spelernaam
    \item Optie F: Bijhouden van uitbetalingen op een losgekoppelde en vervangbare wijze
    \item Optie H: Het spel modelleren als een reeks toestandsovergangen
    \item Optie I: Het aanbieden van insuranceregels
    \item Optie X: Een zelfbedachte spelwijziging met behulp van een design pattern    
\end{itemize}

Vergeet niet ergens op te schrijven (bijvoorbeeld in `design-patterns.md')
welke pattern(s) je hebt gekozen.

\newpage
\section{Klaar? Voorbereiden voor het assessment}
Neem het opdrachtdocument nogmaals door, bekijk de rubric op Canvas en doorloop alle deelopdrachten 
om te zien of je overal aan voldoet. 
Probeer de laatste feedback/les te gebruiken om te kijken of je begrijpt hoe het project in elkaar
zit en op welke manier de theorie daarin terug komt.

Tijdens het assessment zal je met je docent over je project praten.
Het wordt afgesloten met een interview over de theorie achter je project
op basis van de leerdoelen.
Het is handig om een korte presentatie voor te bereiden.

Zorg dat je een assessmentdatum inplant met je docent en bereid je voor door een presentatie of 
een document te maken waarin de theorie en de praktijk van deze cursus samenkomen.

Bedankt voor het volgen van deze cursus en veel succes met het assessment!