\chapter{Design principles}

\section{ICE principles}

\subsection{Program to an Interface, not an Implementation}

\blockquote{
There are two benefits to manipulating objects solely in terms of the interface defined
by abstract classes:
\begin{enumerate}
    \item Clients remain unaware of the specific types of objects they use, 
    as long as the objects adhere to the interface that clients expect.
    \item Clients remain unaware of the classes that implement these objects.
    Clients only know about the (...) [abstraction] defining the interface.
\end{enumerate}
\newline\newline
This so greatly reduces implementation dependencies between subsystems that it leads
to the following principle ofreusable object-oriented design:
\newline\newline
\textit{Program to an interface, not an implementation.}
}{\cite{Gof1994}, p. 18.}

\subsection{Favor Object Composition over Implementation Inheritance}
\blockquote{That leads us to our second principle of object-oriented design:
\newline\newline
\textit{Favor object composition over class inheritance.}
\newline\newline
Ideally, you shouldn't have to create new components to achieve reuse. You should
be able to get all the functionality you need just by assembling existing components
through object composition. But this is rarely the case, because the set of available
componentsis never quite rich enough in practice. Reuseby inheritancemakesit easier
to make new components that can be composed with old ones. Inheritance and object
composition thus work together.
\newline\newline
Nevertheless, our experience isthat designers overuse inheritance as a reuse technique,
and designs are often made more reusable (and simpler) by depending more on object composition.}
{\cite{Gof1994}, p. 20.}

\subsubsection{Encapsulate what varies}


\section{SOLID principles}

\subsection{Single Responsibility Principle (SRP)}
Een manier om de verantwoordelijkheid van een klasse in te richten 
is door te kijken naar de informatie waar deze over kan beschikken. 
Breng die dingen samen die dezelfde informatie moeten 
maken, aanpassen, lezen en verwijderen.

Dit staat ook wel bekend als 
het \emph{information expert-principe}:
\blockquote{
    Assign a responsibility to the information expert---the class 
    that has the \emph{information} necessary to fulfill the responsibility.
}{\cite{Larman2004}, p. 248}

In plaats van allemaal lijsten bij te houden van namen en adressen,
kan je vaak beter een lijst bijhouden met personen, waarin namen en adressen
zijn opgenomen.

\subsection{Open Closed Principle (OCP)}

\subsection{Liskov Substitution Principle (LSP)}

\subsection{Interface Segregation Principle (ISP)}

\subsection{Dependency Inversion Principle (DIP)}
