\chapter{Services en dependency injection}

Een terugkerend thema tot nu toe is dat we onze code flexibel willen houden. In Java kennen we het 
verschil tussen een Interface (wat kan een object allemaal doen), en de Implementatie daarvan (hoe dan?!).

In het hoofdstuk Software Architectuur hebben we lagen en componenten besproken, maar dat hoofdstuk is
voor dit vak optioneel, dus we zullen hier een aantal punten herhalen. 

\section{Services in Software Architectuur}

We hebben onze Software al sinds het begin van de opleiding in lagen ingedeeld. 
De belangrijkste eerste verdeling was tussen "De presentatielaag" en "De rest". 
In een standaard GUI applicatie bepaalt de presentatielaag hoe dingen eruit zien 
(bijv. knopjes, kleuren, schermen etc.), en daar hebben gebruikers gewoon standaard heel veel
(en veranderende!) meningen over.

In een standaard webapplicatie is het ook prima om (op een uitgezoomed niveau) de Frontend als
een presentatielaag te zien. Lange tijd was dat ook de standaard, maar zowel de Frontend applicatie
als de Backend applicatie zijn de afgelopen jaren zoveel complexer geworden (want onze gebruikers worden
veeleisender), dat het standaard is geworden om de Frontend applicatie (met een framework als React, Vue, Angular
of wat dan ook) als geheel aparte applicatie te zien, van de Backend webservice.

Als we dus los naar onze Backend webserivce applicatie kijken, dan heeft die ook een "Presentatielaag", namelijk 
wat voor URLs, parameters en andere HTTP zaken (Headers, Verbs, daarover later allemaal meer) er ondersteund
worden. De @RestControllers (of in oudere frameworks de Servlets) bepalen de buitenkant van de applicatie.

En nog steeds willen we die buitenkant graag loskoppelen van de rest van de applicatie.
Als we een usecase geimplementeerd hebben, zodanig dat deze vanuit een HTTP-Request (dmv. een @RestController)
kan worden aangeroepen, dan is het helemaal niet onredelijk om te verwachten dat er misschien ook
een GUI of commandline applicatie voor Admins is die soortgelijke acties ook kan uitvoeren. Hoe dan ook
is het altijd erg handig om automatische tests te kunnen aansluiten op de implementatie van zo'n usecase, en
als je het begrip een beetje breed interpreteert is een Unit-Test ook gewoon een prima buitenkant (en daarmee
een "Presentatielaag").

Binnen die "De Rest" willen we vaak ook een onderscheid maken tussen Applicatie-logica en Domein-logica.
Met domein-logica bedoelen we alle regels en gekkigheden die optreden binnen het onderwerp waar de applicatie 
over gaat (wanneer is een cijfer voldoende, wanneer komt een klant in aanmerking voor korting, etc.) loskoppelen
van de applicatie-logica (hoe logt een gebruiker in, hoe slaan we onafgeronde bestellingen op, hoe loggen we
errors, etc.). Het verschil tussen domein en applicatielogica is een grijs gebied, en het loopt vaak in elkaar over.
Maak je dus over dat grijze gebied niet teveel zorgen, dan kies je er gewoon eentje op de gok
(en je zult altijd developers tegenkomen die je maar wat graag vertellen dat het \emph{overduidelijk} die andere had moeten zijn). 
Wanneer het winkelmandje van een webwinkel verloopt kun je bijvoorbeeld als zowel domeinlogica als applicatielogica zien.

Die applicatielogica stoppen we graag in een Applicatielaag (ook wel "Taakspecifieke laag" of gewoon "Servicelaag" genoemd).
Deze applicatielaag ligt bovenop de domeinlaag stuurt onze Domeinlaag op verschillende manieren aanstuurt om zo de kern-diensten (services!) van de 
applicatie te realiseren. 

Kortom het is handig om de diensten die onze applicatie aanbiedt als losse dingen te modelleren. Zodat 
we die kunnen combineren op verschillende manieren. Dat is de essentie van het ontwerpen van Services.

Een andere standaard laag die vaak besproken wordt is de Data-laag, daar komen we in het hoofdstuk Persistentie
uitgebreider op terug.

\subsubsection*{Servicelagen, domeinlagen en Unit Tests}

In dit hoofdstuk zeggen we dat de Applicatielaag handig is voor Unit Tests. Terwijl we ook vaak zeggen dat het
belangrijk is om je Domeinlaag te testen!

Hoe zit dat nou?

Het flauwe antwoord is dat beide handig is. Omdat de applicatielaag zo belangrijk is wil je dat daar goede 
tests voor zijn. Maar omdat de applicatielaag vaak erg veel doet is het ook vaak lastig dit heel goed te testen.
We willen bijvoorbeeld voor een webwinkel zeker een paar belangrijke scenarios testen in de applicatielaag 
(kunnen we producten bestellen, wordt alles goed verwerkt, wordt er korting toegepast, etc.).

Maar als we een ingewikkeld systeem hebben waarbij vaste klanten via één of andere berekening korting krijgen,
dan is het vaak welliswaar mogelijk dit te testen via de applicatielaag, maar het is gewoon veel meer werk
dan deze logica direct in de Domeinlaag te testen, omdat je dan alleen de classes hoeft te testen die bij dat
kleine deel van het proces horen.

Het doel van testen is dat je vertrouwen krijgt dat je code foutloos zal werken in een productie-omgeving. Dus
je kiest waar en hoe je het test op basis van een goed compromis tussen dat vertrouwen en ouderwetse gemakzucht\footnote{
    Gemakzucht... niet echt natuurlijk. Wij worden als programmeurs betaald voor werkende features, niet voor 
    werkende tests. Een deel van het kiezen van het juiste testniveau is dus op een professionele manier omgaan
    met "de baas z'n centen".
}.

\subsection{Services in Spring Boot}

Een Service-class in Java lijkt in veel opzichten gewoon op een ordinaire class. Zo heeft het methods
en fields, maar de smaak is vaak een beetje anders. Een Person class heeft misschien een naam, en een 
favoriete smaak ijs, die gedurende het leven van een object regelmatig verandert.

Een pure Service-class is vaak echter wat statiger. Het is vooral een aanspreekplek voor methods, en zodoende zal
er in de fields van een Service vaak (niet zoveel) \emph{data} staan, maar meer andere, meer specialistische
technische dienst-classes, om naar door te kunnen verwijzen.

Je zou het kunnen vergelijken met een soort hoofdaannemer. Jij wil als client gewoon een Huis 
(of een fietsenstalling, of een badkamer, of wat dan ook). De aannemer heeft een netwerk van 
allemaal timmermannen, metselaars, installateurs en stucadoors

Meestal vertaalt een Service class wensen uit de presentatielaag naar de domeinlaag, en stuurt daarbij de meer
technische classes aan. Je domeinlaag bestaat vaak uit POJOs (Plain Old Java Objects), en die technische diensten 
heb je vaak verstopt achter een interface.

Zo heeft de Chips-service in het casino toegang tot een Chips-repository, voor persistentie. Misschien zou
een toekomstige versie nog waarschuwingen kunnen versturen (middels een Notifaction-interface, waarachter dan 
mails, smsjes of wie-weet kan liggen; dat is het mooie van interfaces!).

Kortom een Service class heeft mooie methodes die representeren wat een applicatie allemaal kan, 
en de 

\section{Dependency injection}

Dependency Injection is een ingewikkelde term voor een (achteraf\ldots) vrij simpel idee. Veel
classes, en vooral (maar niet alleen) het soort Service classes zoals we hierboven beschreven
zijn afhankelijk van andere classes om hun werk te doen. Ze hebben dependencies.

De meest voor de hand liggende methode om in een OOP taal als Java aan die dependencies te komen
is om zelf de benodigde constructors aan te roepen.

\begin{listing}[H]
    \begin{minted}[linenos]{java}
    @Service
    public class RegistrationService {
        private final MailChimpMailer mailer;
        private final PostGresRegistrationRepository registrations;
        private final KpnAdresCheckerAPI adresChecker;

        //Geen Dependency Injection
        public RegistrationService(){
            //Wat sommige van die parameters doen... geen idee, dan moet je echt de details in duiken...
            this.mailer = new MailChimpMailer("https://blabla", "some-user-name", "some-password", true, 42);
            this.registrations = new PostGresRegistrationRepository("localhost", "15432", "db-user", "db-password");
            this.adresChecker = new KpnAdresCheckerAPI("https://blabla", "een-of-andere-API-key");
        }

        public void Register(...){
            //En dan doen we hier iets met al die private fields
            ...
        }
    }
    \end{minted}
    \caption{Een voorbeeld zonder Dependency Injection (een "Control Freak").}
    \label{di:controlfreak}
\end{listing}

Zoals we zien in \ref(di:controlfreak) neemt deze RegistrationService class alle verantwoordelijkheid
om alle technische zaken zelf netjes te regelen. Dat heeft als grote voordeel dat het 'lekker makkelijk'
is om zo'n Service te constructen (want geen parameters). Het heeft ook nadelen, want stel
we willen in een andere class rapporteren over de registraties die gedaan zijn, dan moeten we in
code heel goed gaan controleren of daar wel degelijk met dezelfde database geconnect wordt
\footnote{Je zal niet de eerste developer zijn die acceptatiegegevens rapporteert op productie, of vice-versa. 
Dat kan ik je (helaas) garanderen.}.

Verder is deze Service class ook weinig flexibel. Het is nagenoeg onmogelijk om hier bijv. goede
automatische tests voor te schrijven (want je moet op één of andere manier calls naar 2 APIs en een 
echte database zien gaan onderscheppen. Dat kan, maar is niet makkelijk, of leuk).

\begin{listing}[H]
    \begin{minted}[linenos]{java}
    @Service
    public class RegistrationService {
        private final MailChimpMailer mailer;
        private final PostGresRegistrationRepository registrations;
        private final KpnAdresCheckerAPI adresChecker;

        //Dit is technisch gezien Dependency Injection, alleen niet zulke goede
        public RegistrationService(
            MailChimpMailer mailer, 
            PostGresRegistrationRepository registrations,
            KpnAdresCheckerAPI adresChecker
            ){
            this.mailer = mailer;
            this.registrations = registrations;
            this.adresChecker = adresChecker;
        }

        public void Register(...){
            //En dan doen we hier iets met al die private fields
            ...
        }
    }
    \end{minted}
    \caption{Een voorbeeld met matige Dependency Injection.}
    \label{di:baddi}
\end{listing}

Laten we het iets beter maken in \ref{di:baddi}. We hebben hier in elk geval al die configuratie 
weg kunnen duwen (zodat dat netjes op één centrale plek geregeld kan worden).

Dat is mooi, maar het grote probleem hier is dat we met een hele kleine wijziging onszelf veel meer
flexibiliteit kunnen geven. Als we toch al niet verantwoordelijk zijn voor exact hoe bijv. die 
repository geconfigureerd is, willen we dan wel verantwoordelijk zijn dat het altijd exact een PostGres
database moet zijn?

\begin{listing}[H]
    \begin{minted}[linenos]{java}
    @Service
    public class RegistrationService {
        private final Mailer mailer;
        private final RegistrationRepository registrations;
        private final CheckerAPI adresChecker;

        //Nette standaard DI, aangenomen dat deze parameters Java-interfaces zijn
        public RegistrationService(
            Mailer mailer, 
            RegistrationRepository registrations,
            CheckerAPI adresChecker
            ){
            this.mailer = mailer;
            this.registrations = registrations;
            this.adresChecker = adresChecker;
        }

        public void Register(...){
            //En dan doen we hier iets met al die private fields
            ...
        }
    }
    \end{minted}
    \caption{Constructor Injection.}
    \label{di:constructordi}
\end{listing}

Het standaardpatroon zien we vervolgens in \ref{di:constructordi}. Het enige verschil met \ref{di:baddi} 
is dat we alle concrete classes in onze parameters en fields hebben vervangen met Java interfaces.

Dit levert een flexibel ontwerp op, wat (als je er een beetje aan gewend bent), ook dienst doet
als vrij leesbare documentatie. Elke (Service-)constructor gaat een klein, maar netjes lijstje bevatten
aan exact de interfaces die nodig zijn om de rol te vervullen.

Dit heet met een sjiek woord "Constructor Injection", en het is de standaard manier om Dependency Injection
te implementeren (denk 90+\% van de gevallen).

Voor de volledigheid lichten we ook de andere mogelijkheden kort(!) toe. Voor details verwijzen we naar \cite{SeemannDependencyInjection}.

\begin{enumerate}
    \item Constructor Injection
    \item Property Injection
    \item Parameter Injection
    \item Ambient Context
    \item Service Locator (Anti-Pattern!)
\end{enumerate}

Constructor injection betekent dat je alle benodigde dependencies in je constructor vraagt zoals we
gezien hebben. Dit is bijna altijd de juiste optie. Het grote nadeel van Constructor Injection is
dat je het risico loopt gigantisch grote constructors te krijgen. Mijn advies zou zijn om te mikken
op max. 4-6 constructorparameters (een richtlijn, geen wet). 
Als je meer nodig hebt, dan is de kans groot dat jouw class gewoon teveel verschillende dingen 
probeert te doen (de S van SOLID). Hoe dan ook is het goed om wat standaard alternatieven voor
constructor-injection te kennen, want als je alleen maar een hamer hebt, lijkt straks alles op een 
spijker.

Soms is er sprake van Optionele Dependencies (stel het mailtje na een registratie is niet zo boeiend),
dan is het een beetje lelijk om in je constructor een Dependency te vragen, die je misschien helemaal
niet nodig hebt! Dan biedt Property Injection een uitkomst.

\begin{listing}[H]
    \begin{minted}[linenos]{java}
    @Service
    public class RegistrationService {
        private final Mailer mailer;
        private final RegistrationRepository registrations;
        private final CheckerAPI adresChecker;

        public void setMailer(Mailer mailer){
            this.mailer = mailer;
        }

        public RegistrationService(
            RegistrationRepository registrations,
            CheckerAPI adresChecker
            ){
            //We vullen een default-waarde in, anders moeten we straks op Null checken
            //Dan is het veiliger te zorgen dat er altijd -iets- is.
            //Ook al zal deze implementatie, als we gokken adhv. de naam, nooit
            //daadwerkelijk een mailtje sturen.
            this.mailer = new NeverActuallyMailMailer();
            this.registrations = registrations;
            this.adresChecker = adresChecker;
        }

        public void Register(...){
            //En dan doen we hier iets met al die private fields
            ...
        }
    }
    \end{minted}
    \caption{Property Injection.}
    \label{di:propertydi}
\end{listing}

In het voorbeeld \ref{di:propertydi} zien we dat we in de constructor niet meer een Mailer als
parameter vragen. Maar dat we een setter exposen. Wel zorgen we er voor dat er een goede default
implementatie beschikbaar is, door (in dit geval) een NoOp implementatie in ons private field te zetten.
(Dit heet ook wel het Null-Object Design Pattern)

Eerder in de cursus hebben we het gehad over het begrip Cohesion. En in een class met goede cohesion
hebben veel methodes allemaal veel van de private fields nodig. Stel nou dat onze RegistrationService
een method of 15 heeft\ldots en van die 15 methods gebruikt alleen de register method de AdresChecker.
Dat is vanuit het perspectief van Cohesie niet zo handig, want al die andere 14 methods hebben
die dependency niet nodig! En toch vragen we 'm in de constructor. Dat is dus erg onvriendelijk
in het gebruik als je een stuk code wilt schrijven dat tevreden is met alleen die 14 andere methods.

In dat soort gevallen kan Parameter Injection een uitkomst bieden. In plaats van dat we dan 
die technische dependency in de constructor vragen, vragen we 'm alleen als parameter in die ene
method die dat nodig heeft. Dan is de cohesie van onze RegistrationService class beter. Het nadeel
is dat je register method ineens naast allerhande data-parameters (wie registreert zich, waarom, etc.)
ook een technische service als parameter heeft.

Dan blijven er nog twee soorten over: één twijfelachtige, en één gevaarlijke. 
Soms heb je dependencies (bijv. om de werking van je programma te kunnen loggen) echt \emph{overal} nodig. 
Dan is het best irritant om dezelfde dependency (de Logger) in alle constructors van al je classes toe te voegen.

Dit is een geval waarin het oude Singleton-pattern nog een goede rol kan hebben. Er is dan een 
globaal toegankelijke variabele (een public static getter) die je overal kan aanroepen om bijv. je 
applicatie van logging te voorzien. Dit noemt men in DI-literatuur wel een "Ambient Context".

Tot slot is er nog iets dat je zeker weten niet moet doen. En dat stond vroeger bekend als het
Service-Locator (of Service Registry) Design Pattern. In dat Anti-Pattern is er één publiek 
toegankelijke variabele die al je dependencies kan teruggeven. Een soort super-telefoonboek.

Dat is heel krachtig, en voelt (voor kleine applicaties) superhandig. Maar omdat dat telefoonboek
overal toegankelijk is, heb je totaal geen controle meer over waar wat in je applicatie iets gebeurd.
Elke class kan immers op elk moment dat telefoonboek pakken, en alle functionaliteit van de hele applicatie 
aanroepen.

Service Locator \& Control Freaks zijn bijna altijd foute boel. Ambient Contexts zijn twijfelachtig. In de eerste
editie van \cite{SeemannDependencyInjection} is het nog een Design Pattern (goed), in de tweede editie een 
Anti Pattern (slecht). Het grote verschil is dat Ambient Context, je 1 dependency geeft als global variable, maar
Service Locator geeft er heel veel. En global variables zijn vroeg of laat ellende.

\subsubsection*{Dependency Injection \& Inversion of Control}

Er is een lange geschiedenis van verwarring en verwisseling tussen de termen Dependency Injection en Inversion of Control.
Het is handig hiervan op te hoogte te zijn, aangezien je bij het lezen van documentatie of literatuur
er soms rekening mee zult moeten houden dat deze termen verwisseld of door elkaar gebruikt worden.

In deze tekst proberen we vrij consistent met Inversion of Control het idee van 'Control flow'
aan te duiden. Waar normaal gesproken jouw \emph{main} method en daaruit voortvloeiende code de 
globale werking van het programma aanstuurt, is die controle omgedraaid (inverted) als je direct
de controle overdraagt aan bijv. een framework als Spring.

Dit idee zie je ook terug bij bijvoorbeeld het Observer Pattern (denk aan bijv. click-handlers in 
frontend). In plaats van dat we zeggen "dit moet er gebeuren" registreren we een handler bij het click-event.
De controle over wanneer die handler wordt aangeroepen is overgedragen (inverted) aan het frontend-
framwork en de gebruiker.

Tot slot, en daar komt de verwarring vandaan, zie je dit ook terug bij Dependency Injection.
Als onze class niet meer zelf al z'n dependencies new()'t (de "control freak"), maar polymorphistische
interfaces vraagt in diens constructor, dan draagt de class dus feitelijk de controle over 
de exacte werking over aan wie-dan-ook de class moet constructen. Dat kan ordinaire OOP code zijn,
of in het geval van Spring een DI Container.

Kortom, de verwarring is begrijpelijk, maar op moment van schrijven kom je met een zoekterm 
als "Spring IoC Container" nog steeds erg veel waardevolle resultaten tegen. 
Dus is het handig om te weten dat beide termen nog in gebruik zijn.


\subsection{Dependency Injection in Spring (Boot)}

Spring is in de kern een Dependency Injection container. Maar dan wel degene met meer extras dan alle anderen 
(over in elk geval alle programmeertalen waar ik ooit mee gewerkt heb).

Dus wat is een Dependency Injection container? En waarom zouden we er één willen?

Zoals we hierboven hebben gezien is Dependency Injection in de basis niet zo ingewikkeld. Je gebruikt interfaces,
zoals ze bedoeld zijn, en zorgt dat die als parameter een constructor (of setter of method, etc.) in gegooid wordt.
Het probleem begint te onstaan als je dependencies ook weer dependencies hebben. Dan kan er al snel een ingewikkeld 
web van dependencies ontstaan.

\begin{listing}[H]
    \begin{minted}[linenos]{java}
    @SpringBootApplication
    public class HelloApplication {
        public static void main(String[] args){
            SpringApplication.run(HelloApplication.class, args);
        }
    }

    @Component
    public class HelloRunner implements CommandLineRunner {
        public HelloRunner(GreetingService greetingService) {
            this.greetingService = greetingService;
        }

        public void run(String... args){
            this.greetingService.greet("Bob");
        }
    }

    @Service
    public class GreetingService {
        private Greeter greeter;

        public GreetingService(Greeter greeter){
            this.greeter = greeter;
        }

        public void greet(String name){
            String fullGreeting = greeter.createGreeting(name);
            System.out.println(fullGreeting);
        }
    }

    public interface Greeter {
        String createGreeting(String name);
    }

    @Component
    @Primary
    public class PoliteGreeter implements Greeter {
        public String createGreeting(String name){
            return "Greetings " + name;
        } 
    }

    @Component
    public class ImpoliteGreeter implements Greeter {
        public String createGreeting(String name){
            return "Yo " + name;
        } 
    }
    \end{minted}
    \caption{Spring Dependency Injection voor Hello World.}
    \label{di:springexample}
\end{listing}

In \ref{di:springexample} zie je een groter voorbeeld hoe dit allemaal in z'n werk gaat. Uiteraard is het niet
bedoeld als realistisch voorbeeld (niemand doet zoveel moeite voor een Hello-World achtige applicatie), maar het is 
een klein scenario (pastte net op een A4) wat in elk geval de aanpak van Spring laat zien.

We gaan er even vanuit dat dit allemaal in losse files, in een named package zit. Dan zorgt @SpringBootApplication (r1)
er in combinatie met .run() (r4) voor dat alle andere onderdelen op het Component Scan Path zitten (en dus gevonden worden).

Spring ziet dus alle @Component en @Service classes en stopt die in de container, een soort telefoonboek. 
Dat heet in Spring-termen de ApplicationContext (de run() method returnt dit object, mocht je het eens van 
dichterbij willen bekijken). De Spring-naam voor al deze dependencies zijn Beans.

Spring ontdekt dat één van de geregistreerde dependencies een CommandLineRunner is (r9). Dan weet Spring dat die
direct geconstruct moet worden, zodat de run() (r14) method uitgevoerd kan worden. Spring begint dus optimistisch
aan de constructie van deze HelloRunner, en komt snel tot de conclusie dat er een GreetingService gemaakt moet worden.

Vol optimisme begint Spring dan de GreetingService te maken, en komt er (wellicht ontstaat er al enige irritatie
bij de Spring-bot) dat er dan toch eerst een Greeter object gemaakt moet worden. Zo'n Greeter object is echter niet
zomaar te maken, want dat is een \emph{interface}. Spring zal dus mopperend door de lijst van alle geregistreerde 
dependencies lopen om te kijken welke kandidaten die kent voor deze interface.

Stiekem hoopt onze Spring-bot dat er maar één class zal zijn die de interface implementeert, maar helaas voor de
arme container zijn er twee (r39 en r46). Standaard zou Spring nu een foutmelding moeten gooien (want Spring wil
geen onduidelijkheid welke er gekozen moet worden), maar gelukkig staat er @Primary (r38).

\subsubsection*{Spring Boot en @Autowired}

Tot slot moeten we nog heel even iets zeggen over @Autowired, dit komt namelijk héél veel voor in code-voorbeelden
en documentatie op het internet. En het is iets dat je in je eigen code zoveel mogelijk wil vermijden.

We hadden de GreetingService (of de runner) in \ref{di:springexample} ook zo kunnen schrijven als in \ref{di:autowired}.

\begin{listing}[H]
    \begin{minted}[linenos]{java}
    @Service
    public class GreetingService {
        @Autowired
        private Greeter greeter;

        public void greet(String name){
            String fullGreeting = greeter.createGreeting(name);
            System.out.println(fullGreeting);
        }
    }
    \end{minted}
    \caption{@Autowired.}
    \label{di:autowired}
\end{listing}

Dit is 2 regels korter dan het alternatief, maar veel slechter. We kunnen nu ineens geen instantie van deze class
meer maken zonder dat we het Spring framework gebruiken. We kunnen immers niet bij de private-fields van de class.

Dus als we hier ooit iets anders mee zouden willen buiten Spring, dan zitten we gelijk vast. Unit-Tests zijn een 
veel voorkomend geval waarin we graag Spring zoveel mogelijk aan de kant zetten. Spring kost namelijk enkele tientallen
tot honderden milisecondes om op te starten. Als we dat bij elke unit-test gaan doen, dan gaan onze automatische
tests al snel irritant lang duren \footnote{Een beetje 'echte' applicatie heeft al gauw honderden tot duizenden unit-tests}.

Kunnen we @Autowired dan gewoon vergeten? Nee, helaas dat ook niet.
Soms kom je een library of framework tegen waarin een class een parameterloze constructor \emph{moet} hebben.

Ironisch genoeg zijn de Unit-Test classes van JUnit zo'n voorbeeld. Als we daar in een test onze echte database code
zouden willen testen (tegen een Test-database natuurlijk!), dan willen we graag onze database class (bijv. de RegistrationRepository
uit \ref{di:constructordi}) de JUnit test in injecteren. Dat zou er dan als in \ref{di:autowiretest}:

\begin{listing}[H]
    \begin{minted}[linenos]{java}
    @DataJpaTest
    public class RegistrationRepositoryTest {
        @Autowired
        private RegistrationRepository registrations;

        @Test
        public void canSaveRegistrations(){
            ... //hier komen we in het hoofdstuk Persistentie nog wel op terug
        }
    }
    \end{minted}
    \caption{@Autowired kan noodzakelijk zijn in JUnit tests.}
    \label{di:autowiretest}
\end{listing}

Dus in de meeste gevallen zorgt @Autowired voor problemen bij het testen. Maar als je nou juist Spring iets in je test 
wil laten injecteren is het je enige optie. 