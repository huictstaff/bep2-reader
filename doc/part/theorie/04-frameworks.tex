\chapter{Libraries en frameworks}

We hebben het tot nu toe veel gehad over modules en modulariteit. 
Een groot beoogd voordeel van al deze mooie losse legoblokjes
is dat ze \emph{herbruikbaar} zijn. Maar als modulaire code herbruikbaar is, 
dan is het ook hartstikke logisch dat we modules willen (her)gebruiken die door anderen geschreven zijn.

Uiteraard doen we dit ook al, we gebruiken bijvoorbeeld de "standard library" van onze taal 
(met in Java bijvoorbeeld Math.random, of String.format), maar we hebben ook op andere vlakken andermans 
code gebruikt, om bijv. JSON te genereren.

We worden dus afhankelijk van deze externe code om de functionaliteit van onze applicatie te implementeren. 
In vakyargon zijn we dus \emph{coupled} aan onze \emph{externe dependencies}. In principe is dit een goede zaak:
als andere mensen mooie herbruikbare code hebben geschreven scheelt ons dat mooi tijd \& moeite. Maar zoals
altijd kan je hier juist teveel, of te weinig gebuik van maken.

Het te weinig gebruik maakt van externe dependencies noemt men wel het "Not invented here"-syndroom. 
Je wantrouwt alle code die niet uit je eigen bedrijf komt. Dit heeft als grote nadeel dat je heel veel extra
code zelf zal moeten schrijven. En aangezien het niet je kerntaak is om bijv. een JSON-serializer te schrijven
zal je er veel minder tijd aan kunnen besteden, en is de kans dat de kwaliteit hieronder lijdt erg groot.

Aan de andere extreme kant is er het verhaal van \href{https://en.wikipedia.org/wiki/Npm_left-pad_incident}{Leftpad Incident}.
Leftpad was een vrij eenvoudige dependency die simpelweg spaties aan een string toevoegde. Iets dat je in relatief
korte tijd ook zelf nog wel uittypt, maar niet zo snel als dat je het als externe code van de plank kan trekken.
Je bent echter gekoppeld aan je dependencies, dus als die dependencies ineens verdwijnen, of onverwachte wijzigingen doorvoeren,
dan heb je mogelijk een groot probleem. 

Kortom we moeten altijd op zoek gaan naar de juiste balans hierin. De stabiliteit van de dependency is hierin
de belangrijkste eigenschap. Daarmee bedoelen we hoe vaak, en/of hoe groot de wijzigingen zijn die je verwacht
te moeten kunnen accepteren. Hoe stabieler de dependency, hoe minder nadelig het zal zijn om er aan gecoupled te zijn.

Nu zou je kunnen denken "Maar ik hoef toch niet te updaten? Ik kan toch gewoon de oude versie blijven gebruiken?",
en soms is dat ook zo. Echter in veel gevallen zal er vroeg of laat een kritieke bug, of beveiligingslek ontdekt 
worden, die je wel dwingt om te upgraden. Al je dependencies goed up-to-date houden is dus een goede gewoonte.

Binnen deze externe dependencies willen we wel eens onderscheid maken tussen libraries en frameworks. 
Dit is geen super-scherp onderscheid, maar het is nuttig om verschillende perspectieven te hebben. 
Het verschil zit in de 'flow of control', hoe kun je een lijn van acties door je broncode heen trekken. 

\section{Libraries}

Libraries zijn vaak wat kleinere pakketjes code, die één gericht probleem voor je oplossen. Bijvoorbeeld het 
maken van HTTP requests (Requests, Python), het genereren van JSON (Jackson, Java), het uitvullen van Strings (Leftpad, Javascript),
of het maken van complexere berekeningen zoals Annuïteiten (Finmath-lib, Java).

Het belangrijke is dat een library namens jouw code een probleem oplost of makkelijker maakt. Maar uiteindelijk is jouw applicatie nog
nog steeds \emph{de baas}.

Bij het omgaan met libraries is het belangrijk om te beslissen hoe sterk je gecoupled wil zijn aan de library.
Als je rechtstreeks vanuit jouw code een library aanroept, en je doet dat op meerdere plekken 
(stel je een Python applicatie voor die op meerdere plekken Requests gebruikt om APIs aan te roepen), dan zul je 
op al die plaatsen wijzigingen moeten doorvoeren als de library een onverwachte update uitbrengt. Dan is het een
goed idee om in plaats daarvan een eigen class te schrijven, die zelf de library aanroept, zodat je later maar
op één plek evt. dependency-problemen hoeft op te lossen.

Verder is het ook van belang om na te denken wat die library precies vertegenwoordigt. Veel libraries omvatten
de interactie met een extern systeem (zoals een database, een web-API, een barcodescanner, een webcam, etc.). Juist 
dit soort externe systemen wil je vaak extra in de gaten houden, of in (Unit-)tests kunnen vervangen door iets 
dat prettiger te testen is. Hier is het dus ook handig om een eigen class (en evt. een interface) ertussen te schuiven
zodat je flexibel blijft in hoe je applicatie met deze externe systemen omgaat. Zo'n tussenlaagje noemt men ook wel
een \emph{Gateway}.

\section{Frameworks}

Frameworks zijn ook andermans code. Maar in dit geval kiezen we ervoor om de algemene opzet en structuur van 
onze applicatie aan het framework over te laten. Denk bijvoorbeeld aan Java Servlets (BEP1, jaar 1).
Je definiëert een class, zet die op een speciale plek neer, en override de \emph{doGet} methode. En voilá je code
wordt uitgevoerd als er een HTTP-request binnen komt op een bepaald adres.

Je hebt als programmeur totaal geen zicht op de 'applicatie eromheen', alleen op jouw kleine onderdeeltje dat je er 
in hebt geschoven.

Dit is de grote kracht van frameworks, en dat zorgt er voor dat je als beginnende programmeur al direct alle soorten
applicaties ter wereld kan bouwen. Als het werkt tenminste. Als het niet werkt is het verdraaid lastig om er achter 
te komen waarom nou niet.

Het cruciale verschil met een library is dus dat het framework de baas is. De belangrijkste code in de applicatie, 
de code die jij schrijft om de requirements te implementeren, is \emph{niet} de hoofdapplicatie, maar slechts een
handig ingeschoven legoblokje dat aangeroepen wordt door het framework. Dit noemt men in vakyargon \textbf{Inversion of Control}.

\section{Spring \& Spring Boot}

Spring is een bekend Java framework uit 2004. 
In de kern is het een \textbf{Dependency Injection Container}, maar wat dat precies is komen we later op terug.

Spring is het best te zien als een soort universele kapstok voor Java code. Wat voor Java code je ook hebt
(van GUI-applicaties, Webservices en Commandline applicaties tot Database libraries en Crypto-miners) ergens in Spring
is wel een ideaal haakje om die code aan op te hangen. 

Spring Boot (2014) is een framework om je te helpen met die code ophangen. Het probleem van zo'n universele kapstok
werd namelijk (in de jaren tussen 2004 en 2014) dat het steeds ingewikkelder werd om te configureren wat nou precies
waar moest hangen, en hoe.

Spring Boot is een framework bovenop Spring die je helpt met standaard instellingen en zogeheten "Auto-configurations".
Deze Auto-configurations kun je voorstellen als dat Spring Boot automatisch bepaalde stukken code alvast voor je ophangt.
Als je bijv. ergens op een class een \emph{@RestController} annotatie plakt, dan concludeert Spring Boot automatisch dat
je kennelijk een webapplicatie wil maken (en start een webserver), en hangt die class dan op de juiste plek in die server.

\subsection{Entry Point}

Ook al is het framework in principe de baas hoe de applicatie draait. Er is altijd een baas boven baas. 
We zullen helemaal aan het begin de controle aan het framework moeten geven. Met Spring Boot doen we dat zo:

\begin{listing}[H]
    \begin{minted}[linenos]{java}
    @SpringBootApplication
    public class DemoApplication {
        public static void main(String[] args){
            SpringApplication.run(DemoApplication.class, args);
        }
    }
    \end{minted}
    \caption{Minimale opstartcode voor een Spring Boot applicatie.}
    \label{code:springstart}
\end{listing}

Let er op dat deze class netjes in een (named)package zit. Spring gaat er vanuit dat 
al\footnote{Ok, bijna al je code, maar deze randgevallen wil je echt niet opzoeken, geloof me.} 
je code in een named package zit, en dat alle andere code in hetzelfde package, of een subpackage zit.
In iets nettere bewoording: Spring Boot doet een \emph{Component Scan} startend met het package waar 
de @SpringBootApplication class in zit en al diens subpackages.

\subsection{Hotspots}

Uiteraard moet niet alleen het framework opstarten en het zware werk van ons overnemen. Op een gegeven moment
moet het ook de controle weer teruggeven aan onze eigen stukken code.

Één van de meest eenvoudige zaken om mee te beginnen is een zogeheten CommandLineRunner (met @Component erop geplakt). 
Dat is het Spring equivalent van de oude \emph{public static void main}.

\begin{listing}[H]
    \begin{minted}[linenos]{java}
    @Component
    public class StartupRunner implements CommandLineRunner {
        @Override
        public void run(String... args) throws Exception {
            System.out.println("Hello World!");
        }
    }
    \end{minted}
    \caption{Hello World vanuit een System.out in een Spring applicatie.}
    \label{code:springhello}
\end{listing}

We zien in \ref{code:springhello} een aantal onderdelen. @Component zorgt er voor dat Spring zich bewust wordt van
deze class (mits die zich in het path van de \emph{Component Scan} bevindt zoals eerder genoemd).
Vervolgens weet Spring wat er met deze class gedaan moet worden omdat de CommandLineRunner interface geimplementeerd wordt:
deze class zal kennelijk op de Command Line simpelweg gerun'd moeten worden.

Uiteraard is dit complete overkill voor een simpele Hello World applicatie. Maar een CommandLineRunner kan erg
handig zijn om bepaalde opstartlogica te implementeren (bijv. alvast wat dummy-data voor je applicatie).

Een andere veelgebruikt integratie-punt in deze cursus is de RestController (\ref{code:resthello}).

\begin{listing}[H]
    \begin{minted}[linenos]{java}
    @RestController
    public class HelloController {
        @GetMapping("/hello")
        public String hello() {
            return "Hello World!";
        }
    }
    \end{minted}
    \caption{Hello World vanuit een Spring RestController.}
    \label{code:resthello}
\end{listing}

De @RestController annotatie, een soort subclass van @Component, geeft aan dat deze class methods gaat bevatten
die vanuit een HTTP request aangeroepen moeten kunnen gaan worden. De @GetMapping even verderop is zo'n voorbeeld daarvan.
Als deze class gevonden wordt in de \emph{Component Scan}, dan zal Spring een webserver starten (standaard op
\href{http://localhost:8080}{http://localhost:8080}) en daarbinnen een request naar \href{http://localhost:8080/hello}{http://localhost:8080/hello} doorverwijzen naar deze
methode.

\section{Maven}

Een belangrijk punt aan \emph{externe} dependencies is dat ze extern zijn. Oftewel, de broncode van deze 
libraries of frameworks zit niet in onze repository. Vroeger betekende dat goed documenteren welke externe 
dependencies je nodig had, en hopen dat de lezer wist waar deze gevonden en gedownload kunnen worden.

Gelukkig kunnen we (sinds 2004) gebruik maken van Maven, een command-line tool (maar die ook goed in IDEs geïntegreerd is). 
Maven is gebaseerd op het idee van Package Managers van
Operating Systems zoals bijv. \href{https://en.wikipedia.org/wiki/APT_(software)}{APT} voor Linux, 
\href{https://chocolatey.org/}{Chocolatey} voor Windows, of \href{https://brew.sh/}{Homebrew} voor Mac. 
Dat zijn programmas waar je de naam van een library of applicatie opgeeft en de Package Manager installeert het.
Dat is vaak veel handiger dan je browser openen, via google proberen de juiste site te vinden, 
uitvogelen hoe deze website nou weer georganiseerd is, een installer downloaden, 
en op 'next' drukken totdat de installer tevreden is met je inspanningen.

Wat we willen is in een plain-text document (want dat werkt fijn met versiebeheer tools zoals \textit{Git}) aangeven welke
dependencies er precies nodig zijn, en dat deze lijst van dependencies dan volautomatisch gedownload wordt. Maven doet dit door
middel van de 'dependencies' sectie in de \textit{pom.xml} (Pom staat voor Project Object Model\footnote{
    Een niet bijzonder behulpzame naamgeving. Dus voel je vrij om aan het Surinaamse gerecht te denken, dat doe ik ook\ldots }).
In de 'dependencies' sectie van je pom staan de zogeheten \emph{Maven Coordinates} van de externe library die je wilt hebben.
Die coordinates bestaan uit 3 dingen: een groupId (wie heeft het gemaakt), een artifactId (hoe heet de dependency), 
en tot slot een version. Soms wordt dit opgeschreven in XML (\ref{maven:example1}), soms achter elkaar met dubbele punten ertussen:
\textit{org.junit.jupiter:junit-jupiter:5.10.2}. 

Het handige is nu dat als bijv. \textit{junit-jupiter} zelf ook dependencies heeft (je kunt bijv. in IntelliJ CTRL-klikken op Junit in je pom om die te zien), dat
Maven ook automatisch die dependencies gaat downloaden. En als die dependencies dan weer dependencies hebben, dan pakt Maven die ook mee,
enzovoorts enzovoorts. Deze dependencies van je dependencies noemt men ook wel Transitieve Dependencies.

Verder zie je in \ref{maven:example1} dat ons project zelf ook een volledige set Maven-coordinates heeft
(groupId, artifactId en version). Dat is voor nu niet relevant, maar als we ooit ons project zouden willen 
uploaden naar het publieke internet, dan zijn dat de coordinates waarmee anderen ons als dependency
kunnen toevoegen. Voor nu moet het, omdat het nou eenmaal verplicht is in Maven om dit op te geven\footnote{
    De toevoeging SNAPSHOT aan het versienummer is een Maven-gewoonte om aan te geven dat de versie 'onstabiel'
    is. Technisch houdt niets je namelijk tegen om dezelfde versie opnieuw te uploaden, maar met stiekeme
    wijzigingen. Dat zou echter erg verwarrend zijn voor mensen die jouw code gebruiken. Met de toevoeging
    "-SNAPSHOT" geef je expliciet aan dat je nog wel wat wijzigingen van plan bent door te voeren.
}.

\begin{listing}[H]
    \begin{minted}[linenos]{xml}
<project>
    ...
    <groupId>nl.hu.bep2</groupId>
    <artifactId>casino</artifactId>
    <version>0.0.1-SNAPSHOT</version>
    <name>bep2-huland-casino</name>
    ...
    <dependencies>
        <dependency>
            <groupId>org.junit.jupiter</groupId>
            <artifactId>junit-jupiter</artifactId>
            <version>5.10.2</version>
        </dependency>
    </dependencies>
</project>
    \end{minted}
    \caption{Voorbeeld xml dependency in Maven.}
    \label{maven:example1}
\end{listing}

Naast Dependency Management is Maven ook een handige \emph{build-tool}. Vanuit een IDE als \emph{IntelliJ} merk je er weinig
van, maar als je zonder IDE (en zonder Maven) een Java programma zou moeten compileren, dan zou je vroeger 
één voor één \textit{javac Jouw-Class-Hier.java} voor al je bestanden moeten uittypen. 
Maven compileert, met het commando \emph{mvn compile}, automatisch alle bestanden die onder ./src/main/java 
(en nog een paar directories zoals bijv. ./src/test/java) staan.

Twee andere belangrijke commandos om te kennen zijn \emph{mvn verify}, dit draait al je automatische tests, 
en \emph{mvn package}, dit maakt een mooi .Jar (Java Archive) bestand van al je gecompileerde code. Zo'n Jar file is 
stiekem gewoon een .zip bestand (die je dus ook als zip kan uitpakken) met al je gecompileerde .class files.

Maven is een oude, uitgebreide tool en heeft dus nog allerhande andere functionaliteiten (zoals het hosten van een documentatie website), 
en nog vele subtiliteiten (zoals de "dependencymanagement" sectie, die weer totaal anders werkt dan de "dependencies" sectie), en dan
hebben we het nog niet eens over het concept Lifecycles gehad. Maar voor nu weet je genoeg om met Maven te gaan stoeien.

\subsection*{Maven, Dependencies \& DependencyManagement}

We hebben hierboven uitgebreid zitten praten over hoe Maven-coordinates altijd uit drie delen bestaan (GroupId, ArtifactId, Version),
en dan vervolgens ziet het eerste voorbeeld van Spring er uit als in \ref{maven:spring}. Daar staat geen 
version, en toch snapt Maven je? Hoe kan dat?

\begin{listing}[H]
    \begin{minted}[linenos]{xml}
<project>
        ...
    <dependencies>
        <dependency>
            <groupId>org.springframework.boot</groupId>
            <artifactId>spring-boot-starter-web</artifactId>
        </dependency>
    </dependencies>
</project>
    \end{minted}
    \caption{@RestControllers toevoegen aan Spring.}
    \label{maven:spring}
\end{listing}

De truuc zit 'm in een ander stuk van deze pom:

\section{Samenvatting}

We hebben het kort gehad over externe code, en dat je dat grofweg kan verdelen tussen libraries en frameworks.
Bij libraries blijf jij grotendeels 'de baas' over hoe de structuur van de applicatie er uit ziet,
maar bij frameworks kies je er voor om hun standaard-aanpak te volgen. Die standaard aanpak laat zich
uitbreiden door middel van zogeheten "hot spots" (of extension points); plekken waar jij jouw code in het
framework kan "ophangen".

Het framework dat we in dit vak gebruiken is Spring, en we gebruiken Spring Boot om Spring wat makkelijker
te kunnen configureren en opstarten.

Libraries en frameworks zijn vaak geschreven door derden. Dus je hebt een manier nodig om duidelijk aan 
te geven welke je gebruikt, en idealiter ook iets dat ze automatisch binnenhaalt. Dat zorgt er voor
dat jouw code direct uit een Git-repository bruikbaar is, en er niet eerst nog allerlei spul geïnstalleerd
hoeft te worden. Maven is een stuk gereedschap dat dit probleem oplost.