\chapter{Persistentie}

Een webapplicatie is standaard (/idealiter) stateless. Dat betekent dat (wat de backend betreft) elk request helemaal
uit het niets komt. Alle informatie die nodig is om dat request af te handelen moet in dat request zitten.

Als een webapplicatie echt stateless is, betekent dit dat we op elk moment de applicatie zouden moeten
kunnen herstarten, en dat we dan gewoon verder kunnen waar we gebleven waren (er was immers geen toestand
in het geheugen die door de herstart verloren is gegaan).

Dit is voor webapplicaties van belang, omdat we vaak in het echt een beetje willen kunnen sjoemelen
met het opstarten en afsluiten van applicaties. Het internet is namelijk 24/7 open, en iedereen kan 
bij elke website. Als het druk is willen we misschien alle requests over meerdere servers opsplitsen, 
en als we een update uitvoeren willen we de applicatie even afsluiten, updaten, opstarten, en dan direct weer door.

Toch willen we in onze webapplicatie met data om kunnen gaan. Een applicatie die altijd alles vergeet 
heb je niet zo gek veel aan. We willen dus dat na een herstart belangrijke data opgeslagen is geweest.
We willen dat die data blijvend (persistent) is. Meestal gebruiken we een database hiervoor. En meestal 
is dat een "relationele database", zoals bijv. PostGres.

\section{Domeinmodel versus datamodel}



% \subsection{Entity-Relationship Diagrams (ERDs)} ... geen zin in
\subsection{Object-relation impedance mismatch}

\section{Spring JPA en Hibernate}
\subsection{Entities}

\subsubsection*{Ids}

\subsection{Relaties}

\subsubsection*{Cascades}

\subsection{Repositories}

\subsection{Transacties en het belang van goed doortrekken}

